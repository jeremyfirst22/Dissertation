Molecular dynamics (MD) simulations have proven to be incredibly useful for understanding the complex and dynamic interactions between atoms in biomolecular systems. 
The dynamic movement of each atom, for up to tens of thousands of atoms, can be calculated in MD simulations on timescales from femtoseconds up to microseconds. 
While many biological phenomena occur occur in larger systems and on timescales still out of reach of MD simulations, this dynamic range far exceeds those of any experimental technique. 
However, the major disadvantage is, as a theoretical calculation, the simulations must be benchmarked against experimental information to know if they are accurate. 
Often this is structural information.
Emerging relevant biological phenomena, such as biomolecules at surface interfaces or in non-aqueous solvent, are difficult to characterize structurally and are underrepresented in datasets used for benchmarks;
thus the relibility of MD simulations in these more complex environments is uncertain. 
In addition, non-structural properties, such as the electric field in a biomolecular system are often overlooked and the datasets are more scarce, making such calculations unreliable until there is a sufficient dataset to properly paramterize the underlying force fields for that property. 
However, the interpretation of leading experimental electrostatic probes, \pKa{} probes and vibrational Stark effect (VSE) probes, is complicated by their susceptibility to local interactions. 
While calculating electric fields remains a challenge, the accurate modeling of molecular structure can aid in the interpretation of experiments that report electric field, eventually allowing the development of a dataset of electric field measurements to be used to parameterize electrostatic-centered force fields that accurate calculation and prediction of electric fields. 

The work herein investigates the reliability of MD simulations of biomolecules at complex interfaces, demonstrating a protocol to test and validate force fields in complex environments. 
Additionally, we investigate the effect of local interactions on the ability of nitrile spectra to act as VSE probes. 
We establish specific hydrogen bonding as a dominant factor in nitrile vibrational spectra and provide evidence of a control experiment that may be used to diagnose the hydrogen bonding status of the nitrile probe. 
Finally, we provide an example of the importance of electric field measurements (and thus the importance of experiments to diagnose hydrogen bonding to nitriles) by providing evidence that electrostatic contributions control the rate of hydrolysis of p$^{21}$H-Ras, where altered electric fields in the active site change the rate of hydrolysis causing tumorigenesis. 
This work demonstrates the importance of simultaneous investigation \emph{in silico} and \emph{in vitro} into biological phenomena. 
When combined, experiments and calculations compliment each other to provide a mutual feedback loop that increases the molecular level understanding of how the dynamic motion of atoms can cooperatively achieve an observed outcome. 
